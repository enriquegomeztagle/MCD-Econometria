\documentclass[10pt]{article}

\usepackage[utf8]{inputenc}
\usepackage[spanish]{babel}
\decimalpoint
\usepackage{amsmath, amssymb}
\usepackage{xcolor}
\usepackage{geometry}
\geometry{letterpaper, margin=1in}
\usepackage{graphicx}
\usepackage{float}
\usepackage{colortbl}
\usepackage{caption}
\captionsetup{labelfont={color=blue}, textfont={color=blue}}
\usepackage{tocloft}
\renewcommand{\cftsecfont}{\color{blue}}
\renewcommand{\cftsubsecfont}{\color{blue}}
\renewcommand{\cftsubsubsecfont}{\color{blue}}

\renewcommand{\cftsecpagefont}{\color{blue}}
\renewcommand{\cftsubsecpagefont}{\color{blue}}
\renewcommand{\cftsubsubsecpagefont}{\color{blue}}
\renewcommand{\cfttoctitlefont}{\color{blue}\bfseries}

\usepackage[colorlinks=true, linkcolor=blue, urlcolor=blue, citecolor=blue]{hyperref}
%%%%%%%%%%%%%%%%%%%%%%%%%%%%%%%%%%%%%%%%%%%%%%%%%%%%%%%%%%%%%%%%%%%%%%%%%%%%%%%%%%%%%%%%%%%%%%%%%%%%%%%%%%%%%%
\title{Universidad Panamericana \\ Maestría en Ciencia de Datos \\ Econometría \\ \vspace{0.5cm} Tarea RLM}
\author{Enrique Ulises Báez Gómez Tagle}
\date{\today}
%%%%%%%%%%%%%%%%%%%%%%%%%%%%%%%%%%%%%%%%%%%%%%%%%%%%%%%%%%%%%%%%%%%%%%%%%%%%%%%%%%%%%%%%%%%%%%%%%%%%%%%%%%%%%%
\begin{document}
\maketitle

\tableofcontents

\newpage
%%%%%%%%%%%%%%%%%%%%%%%%%%%%%%%%%%%%%%%%%%%%%%%%%%%%%%%%%%%%%%%%%%%%%%%%%%%%%%%%%%%%%%%%%%%%%%%%%%%%%%%%%%%%%%
\section{Pregunta 1}
\paragraph*{a)} Considere los datos de la tabla \ref{tab:datos_pregunta1}.
\begin{table}[H]
    \centering
    \captionsetup{labelfont={color=black}, textfont={color=black}}
    \begin{tabular}{ccc}
        Y & X2 & X3 \\
        \hline
        1 & 1 & 2 \\
        3 & 2 & 1 \\
        8 & 3 & -3 \\
        \hline
    \end{tabular}
    \caption{Datos de la pregunta 1}
    \label{tab:datos_pregunta1}
\end{table}

\paragraph*{b)} Con base en estos datos, estime las siguientes regresiones:
    \[
        Y_i=\alpha_1+\alpha_2X_{2i}+u_{1i},
    \]
    \[
        Y_i=\lambda_1+\lambda_3X_{3i}+u_{2i},
    \]
    \[
        Y_i=\beta_1+\beta_2X_{2i}+\beta_3X_{3i}+u_i,
    \]

    \indent {a)} ¿Es $\alpha_2 = \beta_2$? ¿Por qué? \\

    \indent {b)} ¿Es $\lambda_3 = \beta_3$? ¿Por qué? \\

    \indent {c)} ¿Qué conclusión importante obtiene de este ejercicio? \\  

\textcolor{blue}{
        \textbf{Solución:}  
        Con los datos \((Y, X_2, X_3)\):  
        \[
        Y=\{1,3,8\},\quad X_2=\{1,2,3\},\quad X_3=\{2,1,-3\}.
        \]
        \textbf{Estimaciones:}
        \begin{align*}
        (1)\;& Y_i=\alpha_1+\alpha_2X_{2i}+u_{1i}, & \hat{\alpha}_1=-3,\; \hat{\alpha}_2=3.5. \\
        (2)\;& Y_i=\lambda_1+\lambda_3X_{3i}+u_{2i}, & \hat{\lambda}_1=4,\; \hat{\lambda}_3=-1.3571. \\
        (3)\;& Y_i=\beta_1+\beta_2X_{2i}+\beta_3X_{3i}+u_i, & \hat{\beta}_1=2,\; \hat{\beta}_2=1,\; \hat{\beta}_3=-1.
        \end{align*}
        \textbf{a)} No, \(\alpha_2 \neq \beta_2\). El estimador \(\alpha_2\) en la regresión simple está sesgado porque omite \(X_3\), correlacionado con \(X_2\). Se cumple la fórmula del sesgo por variable omitida.\\
        \textbf{b)} Tampoco, \(\lambda_3 \neq \beta_3\). Análogamente, al omitir \(X_2\), el coeficiente de \(X_3\) se ve afectado por su correlación con \(X_2\).\\
        \textbf{c)} Este ejercicio nos permite entender el \textit{sesgo por variable omitida}. Los coeficientes en regresiones simples (\(\alpha_2,\lambda_3\)) difieren de los verdaderos efectos parciales (\(\beta_2,\beta_3\)) que sólo se identifican en la regresión múltiple.
    }
%%%%%%%%%%%%%%%%%%%%%%%%%%%%%%%%%%%%%%%%%%%%%%%%%%%%%%%%%%%%%%%%%%%%%%%%%%%%%%%%%%%%%%%%%%%%%%%%%%%%%%%%%%%%%%
\section{Pregunta 2}
%%%%%%%%%%%%%%%%%%%%%%%%%%%%%%%%%%%%%%%%%%%%%%%%%%%%%%%%%%%%%%%%%%%%%%%%%%%%%%%%%%%%%%%%%%%%%%%%%%%%%%%%%%%%%%
\section{Pregunta 3}
%%%%%%%%%%%%%%%%%%%%%%%%%%%%%%%%%%%%%%%%%%%%%%%%%%%%%%%%%%%%%%%%%%%%%%%%%%%%%%%%%%%%%%%%%%%%%%%%%%%%%%%%%%%%%%
\section{Pregunta 4}
%%%%%%%%%%%%%%%%%%%%%%%%%%%%%%%%%%%%%%%%%%%%%%%%%%%%%%%%%%%%%%%%%%%%%%%%%%%%%%%%%%%%%%%%%%%%%%%%%%%%%%%%%%%%%%
\section{Pregunta 5}
%%%%%%%%%%%%%%%%%%%%%%%%%%%%%%%%%%%%%%%%%%%%%%%%%%%%%%%%%%%%%%%%%%%%%%%%%%%%%%%%%%%%%%%%%%%%%%%%%%%%%%%%%%%%%%
\section{Link al repositorio con código fuente}
\url{https://github.com/enriquegomeztagle/MCD-Econometria/tree/main/HWs/MLR-Homework}
%%%%%%%%%%%%%%%%%%%%%%%%%%%%%%%%%%%%%%%%%%%%%%%%%%%%%%%%%%%%%%%%%%%%%%%%%%%%%%%%%%%%%%%%%%%%%%%%%%%%%%%%%%%%%%
\end{document}
