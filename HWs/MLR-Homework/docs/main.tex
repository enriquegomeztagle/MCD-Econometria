\documentclass[10pt]{article}

\usepackage[utf8]{inputenc}
\usepackage[spanish]{babel}
\decimalpoint
\usepackage{amsmath, amssymb}
\usepackage{xcolor}
\usepackage{geometry}
\geometry{letterpaper, margin=1in}
\usepackage{graphicx}
\usepackage{float}
\usepackage{colortbl}
\usepackage{caption}
\captionsetup{labelfont={color=blue}, textfont={color=blue}}
\usepackage{tocloft}
\renewcommand{\cftsecfont}{\color{blue}}
\renewcommand{\cftsubsecfont}{\color{blue}}
\renewcommand{\cftsubsubsecfont}{\color{blue}}

\renewcommand{\cftsecpagefont}{\color{blue}}
\renewcommand{\cftsubsecpagefont}{\color{blue}}
\renewcommand{\cftsubsubsecpagefont}{\color{blue}}
\renewcommand{\cfttoctitlefont}{\color{blue}\bfseries}

\usepackage[colorlinks=true, linkcolor=blue, urlcolor=blue, citecolor=blue]{hyperref}
%%%%%%%%%%%%%%%%%%%%%%%%%%%%%%%%%%%%%%%%%%%%%%%%%%%%%%%%%%%%%%%%%%%%%%%%%%%%%%%%%%%%%%%%%%%%%%%%%%%%%%%%%%%%%%
\title{Universidad Panamericana \\ Maestría en Ciencia de Datos \\ Econometría \\ \vspace{0.5cm} Tarea RLM}
\author{Enrique Ulises Báez Gómez Tagle}
\date{\today}
%%%%%%%%%%%%%%%%%%%%%%%%%%%%%%%%%%%%%%%%%%%%%%%%%%%%%%%%%%%%%%%%%%%%%%%%%%%%%%%%%%%%%%%%%%%%%%%%%%%%%%%%%%%%%%
\begin{document}
\maketitle

\tableofcontents

\newpage
%%%%%%%%%%%%%%%%%%%%%%%%%%%%%%%%%%%%%%%%%%%%%%%%%%%%%%%%%%%%%%%%%%%%%%%%%%%%%%%%%%%%%%%%%%%%%%%%%%%%%%%%%%%%%%
\section{Pregunta 1}
\paragraph*{a)} Considere los datos de la tabla \ref{tab:datos_pregunta1}.
\begin{table}[H]
    \centering
    \captionsetup{labelfont={color=black}, textfont={color=black}}
    \begin{tabular}{ccc}
        Y & X2 & X3 \\
        \hline
        1 & 1 & 2 \\
        3 & 2 & 1 \\
        8 & 3 & -3 \\
        \hline
    \end{tabular}
    \caption{Datos de la pregunta 1}
    \label{tab:datos_pregunta1}
\end{table}

\paragraph*{b)} Con base en estos datos, estime las siguientes regresiones:
    \[
        Y_i=\alpha_1+\alpha_2X_{2i}+u_{1i},
    \]
    \[
        Y_i=\lambda_1+\lambda_3X_{3i}+u_{2i},
    \]
    \[
        Y_i=\beta_1+\beta_2X_{2i}+\beta_3X_{3i}+u_i,
    \]

    \indent {a)} ¿Es $\alpha_2 = \beta_2$? ¿Por qué? \\

    \indent {b)} ¿Es $\lambda_3 = \beta_3$? ¿Por qué? \\

    \indent {c)} ¿Qué conclusión importante obtiene de este ejercicio? \\  

\textcolor{blue}{ 
    Con los datos \((Y, X_2, X_3)\):  
    \[
    Y=\{1,3,8\},\quad X_2=\{1,2,3\},\quad X_3=\{2,1,-3\}.
    \]
    \textbf{Estimaciones:}
    \begin{align*}
    (1)\;& Y_i=\alpha_1+\alpha_2X_{2i}+u_{1i}, & \hat{\alpha}_1=-3,\; \hat{\alpha}_2=3.5. \\
    (2)\;& Y_i=\lambda_1+\lambda_3X_{3i}+u_{2i}, & \hat{\lambda}_1=4,\; \hat{\lambda}_3=-1.3571. \\
    (3)\;& Y_i=\beta_1+\beta_2X_{2i}+\beta_3X_{3i}+u_i, & \hat{\beta}_1=2,\; \hat{\beta}_2=1,\; \hat{\beta}_3=-1.
    \end{align*}
    \textbf{a)} No, \(\alpha_2 \neq \beta_2\). El estimador \(\alpha_2\) en la regresión simple está sesgado porque omite \(X_3\), correlacionado con \(X_2\). Se cumple la fórmula del sesgo por variable omitida.\\
    \textbf{b)} Tampoco, \(\lambda_3 \neq \beta_3\). Análogamente, al omitir \(X_2\), el coeficiente de \(X_3\) se ve afectado por su correlación con \(X_2\).\\
    \textbf{c)} Este ejercicio nos permite entender el \textit{sesgo por variable omitida}. Los coeficientes en regresiones simples (\(\alpha_2,\lambda_3\)) difieren de los verdaderos efectos parciales (\(\beta_2,\beta_3\)) que sólo se identifican en la regresión múltiple.
    }
%%%%%%%%%%%%%%%%%%%%%%%%%%%%%%%%%%%%%%%%%%%%%%%%%%%%%%%%%%%%%%%%%%%%%%%%%%%%%%%%%%%%%%%%%%%%%%%%%%%%%%%%%%%%%%
\newpage
\section{Pregunta 2}
\paragraph*{a)} La demanda de rosas. En la Tabla~\ref{tab:demanda_rosas} se presentan datos trimestrales (1971-III a 1975-II) sobre estas variables:\\
$Y$ = cantidad de rosas vendidas (docenas);\\
$X_2$ = precio promedio al mayoreo de rosas (\$/docena);\\
$X_3$ = precio promedio al mayoreo de claveles (\$/docena);\\
$X_4$ = ingreso familiar disponible promedio semanal (\$/semana);\\
$X_5$ = variable de tendencia que toma valores de (1,2,\dots), durante el periodo 1971-III a 1975-II en el área metropolitana de Detroit.\\

\begin{table}[H]
    \centering
    \captionsetup{labelfont={color=black}, textfont={color=black}}
    \begin{tabular}{lccccc}
        \hline
        \textbf{Año-trim} & $Y$ & $X_2$ & $X_3$ & $X_4$ & $X_5$ \\
        \hline
        1971-III & 11484 & 2.26 & 3.49 & 158.11 & 1 \\
        1971-IV  &  9348 & 2.54 & 2.85 & 173.36 & 2 \\
        1972-I   &  8429 & 3.07 & 4.06 & 165.26 & 3 \\
        1972-II  & 10079 & 2.91 & 3.64 & 172.92 & 4 \\
        1972-III &  9240 & 2.73 & 3.21 & 178.46 & 5 \\
        1972-IV  &  8862 & 2.77 & 3.66 & 198.62 & 6 \\
        1973-I   &  6216 & 3.59 & 3.76 & 186.28 & 7 \\
        1973-II  &  8253 & 3.23 & 3.49 & 188.98 & 8 \\
        1973-III &  8038 & 2.60 & 3.13 & 180.49 & 9 \\
        1973-IV  &  7476 & 2.89 & 3.20 & 183.33 & 10 \\
        1974-I   &  5911 & 3.77 & 3.65 & 181.87 & 11 \\
        1974-II  &  7950 & 3.64 & 3.60 & 185.00 & 12 \\
        1974-III &  6134 & 2.82 & 2.94 & 184.00 & 13 \\
        1974-IV  &  5868 & 2.96 & 3.12 & 188.20 & 14 \\
        1975-I   &  3160 & 4.24 & 3.58 & 175.67 & 15 \\
        1975-II  &  5872 & 3.69 & 3.53 & 188.00 & 16 \\
        \hline
    \end{tabular}
    \caption{Demanda trimestral de rosas en Detroit (1971-III a 1975-II).}
    \label{tab:demanda_rosas}
\end{table}

Se le pide considderar las siguientes funciones de demanda:
\begin{align*}
    Y_t &= \alpha_1 + \alpha_2 X_{2t} + \alpha_3 X_{3t} + \alpha_4 X_{4t} + \alpha_5 X_{5t} + u_t, \\
    \ln Y_t &= \beta_1 + \beta_2 \ln X_{2t} + \beta_3 \ln X_{3t} + \beta_4 \ln X_{4t} + \beta_5 X_{5t} + u_t.
\end{align*}

\begin{enumerate}
    \item[\textbf{(a)}] Estime los parámetros del modelo lineal e interprete los resultados.\\
    \textcolor{blue}{
        Modelo lineal estimado por MCO (con intercepto):
        \[
        \widehat{Y}_t= \hat{\alpha}_1 + \hat{\alpha}_2 X_{2t}+\hat{\alpha}_3 X_{3t}+\hat{\alpha}_4 X_{4t}+\hat{\alpha}_5 X_{5t},
        \]
        con $ \hat{\alpha}_1=10820.0,\quad \hat{\alpha}_2=-2227.70\ (t=-2.42,\ p=0.034),\quad \hat{\alpha}_3=1251.14\ (t=1.08,\ p=0.303),$
        \[
        \hat{\alpha}_4=6.283\ (t=0.21,\ p=0.841),\quad \hat{\alpha}_5=-197.40\ (t=-1.94,\ p=0.078).
        \]
        Ajuste e inferencia: \(R^2=0.835\), \(R^2_{adj}=0.775\), \(F=13.89\) (\(p=0.000281\)). Durbin–Watson \(=2.33\).\\
        El precio propio de las rosas (\(X_2\)) tiene signo negativo y es significativo al 5\%; el precio de los claveles (\(X_3\)) es positivo pero no significativo; el ingreso (\(X_4\)) es positivo pero no significativo; la tendencia (\(X_5\)) es negativa y significativa (10\%). El número de condición elevado (\(~4.48\times10^3\)) podría indicar \textit{multicolinealidad} potencial.
    }
    \item[\textbf{(b)}] Estime los parámetros del modelo log-lineal e interprete los resultados.\\
    \textcolor{blue}{
        Modelo log-lineal estimado por MCO:
        \[
        \widehat{\ln Y}_t= \hat{\beta}_1 + \hat{\beta}_2 \ln X_{2t}+\hat{\beta}_3 \ln X_{3t}+\hat{\beta}_4 \ln X_{4t}+\hat{\beta}_5 X_{5t},
        \]
        con $\hat{\beta}_1=3.572,\quad \hat{\beta}_2=-1.1707\ (t=-2.40,\ p=0.035),\quad \hat{\beta}_3=0.7379\ (t=1.13,\ p=0.282), $
        \[
        \hat{\beta}_4=1.1532\ (t=1.28,\ p=0.227),\quad \hat{\beta}_5=-0.0301\ (t=-1.83,\ p=0.094).
        \]
        Ajuste e inferencia: \(R^2=0.799\), \(R^2_{adj}=0.726\), \(F=10.92\) (\(p=0.000798\)). Durbin–Watson \(=2.05\).\\
        Aquí, los coeficientes \(\beta_2,\beta_3,\beta_4\) son elasticidades: la demanda es elástica al precio propio (\(-1.17\), significativo al 5\%), presenta elasticidad cruzada positiva frente al precio de claveles (0.74, no significativa) y es normal (elasticidad ingreso 1.15, no significativa). La tendencia es levemente decreciente (10\%).
        }

    \item[\textbf{(c)}] $\beta_2$, $\beta_3$ y $\beta_4$ dan, respectivamente, las elasticidades de la demanda respecto al precio propio, precio cruzado e ingreso. ¿Cuáles son, a priori, los signos esperados de estas elasticidades? ¿Concuerdan estos resultados con las expectativas a priori?\\
    \textcolor{blue}{
        Expectativas a priori: \(\beta_2<0\) (ley de la demanda), \(\beta_3>0\) si claveles son sustitutos, y \(\beta_4>0\) si las rosas son normal.\\
        \textit{Resultados}: \(\hat{\beta}_2=-1.1707<0\), \(\hat{\beta}_3=0.7379>0\), \(\hat{\beta}_4=1.1532>0\). Los signos \emph{concuerdan} con la teoría; sólo el efecto de precio propio es estadísticamente significativo al 5\%.
        }

    \item[\textbf{(d)}] ¿Cómo calcularía las elasticidades precio propio, precio cruzado e ingreso en el modelo lineal?\\
    \textcolor{blue}{
        Calculamos las elasticidades en un punto de evaluacióncomo
        \[
        \varepsilon_{Y,X_j}= \frac{\partial Y}{\partial X_j}\,\frac{\bar X_j}{\bar Y}=\hat{\alpha}_j\,\frac{\bar X_j}{\bar Y},\quad j\in\{2,3,4\}.
        \]
        Evaluadas en las medias, se obtienen:
        \[
        \varepsilon_{\text{precio propio}}=-0.9053,\quad \varepsilon_{\text{precio cruzado}}=0.5616,\quad \varepsilon_{\text{ingreso}}=0.1484.
        \]
        Para el modelo log-lineal, las elasticidades son constantes e iguales a los coeficientes: \(\varepsilon_{p}=-1.1707\), \(\varepsilon_{pc}=0.7379\), \(\varepsilon_{y}=1.1532\).
    }
    
    \item[\textbf{(e)}] Con base en su análisis, ¿cuál modelo, si existe, escogería y por qué?\\
    \textcolor{blue}{
        Comparación: el modelo lineal exhibe mayor \(R^2_{adj}=0.775\) que el log-lineal (0.726), pero el log-lineal se ve favorecido por los criterios de información (AIC = \(-9.08\), BIC = \(-5.22\) frente a 269.48 y 273.34). Además, el log-lineal entrega elasticidades directamente interpretables y suele capturar mejor relaciones proporcionales.\\
        Con base en AIC/BIC de OLS bajo normalidad y en la interpretación económica (elasticidades), \textbf{elegimos el modelo log-lineal}. No obstante, la muestra es pequeña (\(n=16\)) y hay indicios de multicolinealidad; es recomendable revisar los resultados con precaución.
    }
\end{enumerate}
%%%%%%%%%%%%%%%%%%%%%%%%%%%%%%%%%%%%%%%%%%%%%%%%%%%%%%%%%%%%%%%%%%%%%%%%%%%%%%%%%%%%%%%%%%%%%%%%%%%%%%%%%%%%%%
\section{Pregunta 3}

\paragraph*{a)} Desembolsos del presupuesto de defensa de Estados Unidos, 1962--1981. Para explicar el presupuesto de defensa, considere el siguiente modelo:
\[
Y_t = \beta_1 + \beta_2 X_{2t} + \beta_3 X_{3t} + \beta_4 X_{4t} + \beta_5 X_{5t} + u_t.
\]
Donde: \\
$Y_t$ = desembolsos del presupuesto de defensa durante el año $t$, \$ miles de millones.\\
$X_{2t}$ = PNB durante el año $t$, \$ miles de millones.\\
$X_{3t}$ = ventas militares de Estados Unidos/ayuda en el año $t$, \$ miles de millones.\\
$X_{4t}$ = ventas de la industria aeroespacial, \$ miles de millones.\\
$X_{5t}$ = conflictos militares que implican a más de 100\,000 soldados. Esta variable adquiere el valor de 1 cuando
participan 100 000 soldados o más, y es igual a cero cuando el número de soldados no llega a 100 000. 

\begin{table}[H]
    \centering
    \captionsetup{labelfont={color=black}, textfont={color=black}}
    \begin{tabular}{lccccc}
        \hline
        \textbf{Año} & $Y$ & $X_2$ & $X_3$ & $X_4$ & $X_5$ \\
        \hline
        1962 &  51.1 &  560.3 &  0.6  & 16.0 & 0 \\
        1963 &  52.3 &  590.5 &  0.9  & 16.4 & 0 \\
        1964 &  53.6 &  632.4 &  1.1  & 16.7 & 0 \\
        1965 &  49.6 &  684.9 &  1.4  & 17.0 & 1 \\
        1966 &  56.8 &  749.9 &  1.6  & 20.2 & 1 \\
        1967 &  70.1 &  793.0 &  1.0  & 23.1 & 1 \\
        1968 &  80.5 &  865.0 &  0.8  & 25.6 & 1 \\
        1969 &  81.2 &  931.4 &  1.5  & 24.6 & 1 \\
        1970 &  80.3 &  992.7 &  1.0  & 24.8 & 1 \\
        1971 &  77.7 & 1077.6 &  1.5  & 27.1 & 1 \\
        1972 &  78.3 & 1185.9 &  2.95 & 21.5 & 1 \\
        1973 &  74.5 & 1326.4 &  4.8  & 24.3 & 0 \\
        1974 &  77.8 & 1434.2 & 10.3  & 26.8 & 0 \\
        1975 &  85.6 & 1549.2 & 16.0  & 29.5 & 0 \\
        1976 &  89.4 & 1748.0 & 14.7  & 30.4 & 0 \\
        1977 &  97.5 & 1918.3 &  8.3  & 33.3 & 0 \\
        1978 & 105.2 & 2163.9 & 11.0  & 38.0 & 0 \\
        1979 & 117.7 & 2417.8 & 13.0  & 46.2 & 0 \\
        1980 & 135.9 & 2633.1 & 15.3  & 57.6 & 0 \\
        1981 & 162.1 & 2937.7 & 18.0  & 68.9 & 0 \\
        \hline
    \end{tabular}
    \caption{EE.\,UU.: Presupuesto de defensa y variables explicativas (1962--1981).}
    \label{tab:defensa_usa}
\end{table}

\paragraph*{b)} Con base en la Tabla~\ref{tab:defensa_usa}, responda:
\begin{enumerate}
    \item[\textbf{(a)}] Estime los parámetros del modelo lineal y sus errores estándar, y obtenga $R^2$ y $R^2$ ajustada.\\
    \textcolor{blue}{
        El modelo estimado por MCO (entre paréntesis se reportan los errores estándar) es
        \[
        \widehat{Y}_t=\hat{\beta}_1+\hat{\beta}_2X_{2t}+\hat{\beta}_3X_{3t}+\hat{\beta}_4X_{4t}+\hat{\beta}_5X_{5t},
        \]
        con
        $\hat{\beta}_1=19.7122\ (3.3509),\quad \hat{\beta}_2=0.0164\ (0.0065),\quad \hat{\beta}_3=-0.2261\ (0.4556),\\
        \hat{\beta}_4=1.3967\ (0.2608),\quad \hat{\beta}_5=5.3564\ (3.0201),$ \\
        \\
        Métricas de ajuste: \(R^2=0.9784\), \(R^2_{adj}=0.9726\), \(F=169.5\) (\(p=2.73\times10^{-12}\)). Durbin–Watson = 1.169.\\
        \textit{Ecuación en niveles}: \(\widehat{Y}_t=19.7122+0.0164\,X_{2t}-0.2261\,X_{3t}+1.3967\,X_{4t}+5.3564\,X_{5t}.\)
    }

    \item[\textbf{(b)}] Comente los resultados, considerando cualquier expectativa \textit{a priori}  que tenga sobre la relación entre $Y$ y las diversas variables $X$.\\
    \textcolor{blue}{
        \emph{Signos esperados}: se anticipa efecto positivo de PNB (\(X_2\)), ventas militares/ayuda (\(X_3\)), ventas aeroespaciales (\(X_4\)) y de la dummy de conflicto (\(X_5\)).\\
        \emph{Resultados}: \(\hat{\beta}_2>0\) y significativo al 5\% (\(t=2.51\)); \(\hat{\beta}_4>0\) y altamente significativo (\(t=5.36\)); \(\hat{\beta}_5>0\) y marginal al 10\% (\(t=1.77\)); \(\hat{\beta}_3<0\) y no significativo. \\
        En promedio, manteniendo todo lo demás constante:
        \begin{itemize}
        \item Un aumento de \$1 mil millones en el PNB se asocia con \(0.016\) mil millones adicionales en defensa.
        \item Un aumento de \$1 mil millón en ventas aeroespaciales se asocia con \(1.397\) mil millones adicionales en defensa.
        \item La presencia de un conflicto \(X_5=1\) eleva el gasto en \(\approx 5.36\) mil millones.
        \end{itemize}
        \emph{Diagnóstico}: el número de condición (\(\approx 5.53\times10^3\)) y VIF (sin intercepto) altos obtenidos (p.ej. \(\text{VIF}_{X_2}\approx 80\), \(\text{VIF}_{X_4}\approx 62\)) sugieren  una \textit{multicolinealidad} severa entre regresores macro, lo cual puede inflar errores estándar y volver inestables algunos signos (como \(X_3\)). Durbin-Watson \(=1.17\) sugiere posible autocorrelación positiva de primer orden en residuos (series anuales).
    }

    \item[\textbf{(c)}] ¿Qué otra(s) variable(s) incluiría en el modelo y por qué?\\
    \textcolor{blue}{
        Incluiría variables para (i) trabajar en términos reales y (ii) capturar dinámica/geopolítica:
        \begin{itemize}
        \item Deflactor del gasto de defensa o CPI (para expresar todas las series en términos reales) y una tendencia temporal.
        \item Petróleo y/o choques energéticos 1973–79; tasa de inflación o interés (política macro).
        \item Gasto/PNB rezagado o \(Y_{t-1}\) (inercia presupuestal) y rezagos de \(X_2, X_4\).
        \item Dummies geopolíticas (p.ej., Vietnam 1965–73) o un indicador de tensiones internacionales adicional a \(X_5\).
        \end{itemize}
        Con estas variables, se podrían mitigar sesgos por omisión y reducir la multicolinealidad al separar tendencias comunes entre \(X_2\) y \(X_4\).
    }
\end{enumerate}
%%%%%%%%%%%%%%%%%%%%%%%%%%%%%%%%%%%%%%%%%%%%%%%%%%%%%%%%%%%%%%%%%%%%%%%%%%%%%%%%%%%%%%%%%%%%%%%%%%%%%%%%%%%%%%

\section{Pregunta 4}
La tabla 7.12 presenta datos del gasto de consumo real, ingreso real, riqueza real y tasas de interés reales
de Estados Unidos de 1947 a 2000.
\begin{table}[H]
    \centering
    \tiny
    \captionsetup{labelfont={color=black}, textfont={color=black}}
    \begin{tabular}{lcccc}
        \hline
        \textbf{Año} & \textbf{C} & \textbf{Yd} & \textbf{Riqueza} & \textbf{Tasa de Interés} \\
        \hline
        1947 &  976.4 & 1035.2 &  5166.8 & -10.351 \\
        1948 &  998.1 & 1090.0 &  5280.8 &  -4.720 \\
        1949 & 1025.3 & 1095.6 &  5607.4 &   1.044 \\
        1950 & 1090.9 & 1192.7 &  5759.5 &   0.407 \\
        1951 & 1107.7 & 1227.0 &  6081.6 &  -5.283 \\
        1952 & 1142.4 & 1266.8 &  6243.9 &  -0.277 \\
        1953 & 1221.4 & 1327.5 &  6355.6 &   0.561 \\
        1954 & 1277.2 & 1344.0 &  6797.4 &  -0.138 \\
        1955 & 1314.0 & 1433.8 &  7172.2 &   0.262 \\
        1956 & 1348.8 & 1502.3 &  7375.2 &  -0.736 \\
        1957 & 1381.8 & 1539.5 &  7315.3 &  -0.261 \\
        1958 & 1393.0 & 1553.7 &  7870.0 &  -0.575 \\
        1959 & 1470.7 & 1623.8 &  8188.1 &   2.296 \\
        1960 & 1516.0 & 1664.8 &  8351.8 &   1.511 \\
        1961 & 1541.2 & 1720.0 &  8971.9 &   1.296 \\
        1962 & 1617.3 & 1803.5 &  9091.5 &   1.396 \\
        1963 & 1684.8 & 1871.5 &  9436.1 &   2.085 \\
        1964 & 1784.8 & 2006.9 & 10004.4 &   2.027 \\
        1965 & 1897.6 & 2131.0 & 10562.8 &   2.112 \\
        1966 & 2066.2 & 2244.6 & 11502.0 &   2.220 \\
        1967 & 2066.2 & 2340.5 & 12341.0 &   2.120 \\
        1968 & 2264.8 & 2448.2 & 12145.4 &   1.055 \\
        1969 & 2314.5 & 2524.3 & 11672.3 &   1.732 \\
        1970 & 2405.2 & 2630.0 & 11650.8 &   1.176 \\
        1971 & 2505.5 & 2745.3 & 12312.9 &  -0.712 \\
        1972 & 2650.5 & 2874.3 & 13499.9 &  -0.156 \\
        1973 & 2675.9 & 3072.3 & 13081.0 &   1.414 \\
        1974 & 2653.7 & 3051.9 & 11868.8 &  -1.043 \\
        1975 & 2710.9 & 3108.5 & 12634.4 &  -3.534 \\
        1976 & 2868.9 & 3243.5 & 13456.8 &  -0.657 \\
        1977 & 2992.1 & 3360.7 & 13786.3 &  -1.190 \\
        1978 & 3124.7 & 3527.5 & 14450.5 &   0.113 \\
        1979 & 3203.2 & 3628.6 & 15340.0 &   1.704 \\
        1980 & 3193.0 & 3658.0 & 15965.0 &   2.298 \\
        1981 & 3236.0 & 3741.1 & 15965.0 &   4.704 \\
        1982 & 3275.5 & 3791.7 & 16312.5 &   4.449 \\
        1983 & 3454.3 & 3906.9 & 16944.8 &   5.691 \\
        1984 & 3640.6 & 4207.6 & 17526.7 &   5.848 \\
        1985 & 3820.9 & 4347.8 & 19068.3 &   4.331 \\
        1986 & 3981.2 & 4486.6 & 20530.0 &   3.768 \\
        1987 & 4113.4 & 4586.5 & 21235.7 &   2.819 \\
        1988 & 4279.5 & 4784.1 & 22332.0 &   3.287 \\
        1989 & 4393.7 & 4906.5 & 23659.8 &   4.318 \\
        1990 & 4474.5 & 5014.2 & 23105.1 &   3.595 \\
        1991 & 4466.6 & 5033.0 & 24050.2 &   1.803 \\
        1992 & 4594.5 & 5189.3 & 24418.2 &   1.007 \\
        1993 & 4748.9 & 5261.3 & 25092.3 &   0.625 \\
        1994 & 4928.1 & 5397.2 & 25218.6 &   2.206 \\
        1995 & 5075.6 & 5539.1 & 27439.7 &   3.333 \\
        1996 & 5237.5 & 5677.7 & 29448.2 &   3.083 \\
        1997 & 5423.9 & 5854.5 & 32664.1 &   3.120 \\
        1998 & 5683.7 & 6168.6 & 35887.0 &   3.584 \\
        1999 & 5968.4 & 6320.0 & 39591.3 &   3.245 \\
        2000 & 6257.8 & 6539.2 & 38167.7 &   3.576 \\
        \hline
    \end{tabular}
    \caption{Gasto de consumo real, ingreso real, riqueza real y tasas de interés reales de Estados Unidos (1947--2000).}
    \label{tab:consumo_usa}
\end{table}

\begin{enumerate}
    \item[\textbf{(a)}] Con los datos de la tabla, estime la función de consumo lineal usando los datos de ingreso, riqueza y tasa
    de interés. ¿Cuál es la ecuación ajustada?\\
    \textcolor{blue}{
    Modelo estimado por MCO (errores estándar entre paréntesis):
    \[
    \widehat{C}_t=\hat\gamma_1+\hat\gamma_2\,Yd_t+\hat\gamma_3\,Riqueza_t+\hat\gamma_4\,Tasa_t,
    \]
    con $ \hat\gamma_1=-3.0103\,(15.014),\quad \hat\gamma_2=0.7344\,(0.0160),\quad \hat\gamma_3=0.0354\,(0.0030),\quad \hat\gamma_4=-5.7072\,(2.679).$ \\
    Métricas de ajuste: \(R^2=0.9992\), \(R^2_{adj}=0.9991\), \(F=1.975\times10^{4}\) (\(p=8.05\times10^{-77}\)), \(n=54\). Durbin–Watson = 1.313.\\
    \textit{Ecuación: \(\widehat{C}_t=-3.010+0.7344\,Yd_t+0.0354\,Riqueza_t-5.7072\,Tasa_t\).}
    }
    \item[\textbf{(b)}] ¿Qué indican los coeficientes estimados sobre las relaciones entre las variables y el gasto de consumo?\\
    \textcolor{blue}{ 
        \emph{Signos esperados vs. estimados}:  
        \(\hat\gamma_2>0\) (cumple), \(\hat\gamma_3>0\) (cumple) y \(\hat\gamma_4<0\) (cumple).\\
        \emph{Interpretación marginal}:  
        \begin{itemize}
        \item Propensión marginal a consumir del ingreso disponible: \(\partial C/\partial Yd=\hat\gamma_2=0.7344\). Un incremento de 1 unidad en \(Yd\) aumenta el consumo en 0.734 unidades, ceteris paribus.
        \item Efecto riqueza: \(\partial C/\partial W=\hat\gamma_3=0.0354\). El consumo crece con la riqueza, aunque el impacto unitario es menor que el del ingreso.
        \item Efecto de la tasa real: \(\partial C/\partial r=\hat\gamma_4=-5.7072\). Tasas más altas reducen el consumo (sustitución intertemporal); el coeficiente es significativo al 5\%.
        \end{itemize}
        \emph{Magnitudes relativas (betas estandarizados)}: ingreso \(0.802\), riqueza \(0.209\), tasa \(-0.011\): el ingreso explica la mayor parte de la variación contemporánea del consumo, seguido por la riqueza; la tasa tiene efecto pequeño pero de signo teórico.
        \emph{Multicolinealidad - VIF (sin intercepto) }: \(\text{VIF}_{Yd}\approx 65.9\), \(\text{VIF}_{Riqueza}\approx 65.8\), \(\text{VIF}_{Tasa}\approx 1.56\) y un número de condición \(\approx 4.48\times10^{4}\) sugieren fuerte colinealidad entre ingreso y riqueza; esto puede inflar errores estándar y volver sensibles las estimaciones a la especificación. Para robustez, puede considerarse trabajar en logaritmos/tasas de crecimiento, usar componentes permanentes o rezagos.
    }
\end{enumerate}
%%%%%%%%%%%%%%%%%%%%%%%%%%%%%%%%%%%%%%%%%%%%%%%%%%%%%%%%%%%%%%%%%%%%%%%%%%%%%%%%%%%%%%%%%%%%%%%%%%%%%%%%%%%%%%
\section{Link al repositorio con código fuente}
\url{https://github.com/enriquegomeztagle/MCD-Econometria/tree/main/HWs/MLR-Homework}
%%%%%%%%%%%%%%%%%%%%%%%%%%%%%%%%%%%%%%%%%%%%%%%%%%%%%%%%%%%%%%%%%%%%%%%%%%%%%%%%%%%%%%%%%%%%%%%%%%%%%%%%%%%%%%
\end{document}
