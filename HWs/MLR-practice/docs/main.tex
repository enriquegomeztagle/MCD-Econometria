\documentclass[10pt]{article}

\usepackage[utf8]{inputenc}
\usepackage[spanish]{babel}
\decimalpoint
\usepackage{amsmath, amssymb}
\usepackage{xcolor}
\usepackage{geometry}
\geometry{letterpaper, margin=1in}
\usepackage{graphicx}
\usepackage{hyperref}
\usepackage{float}
\usepackage{colortbl}
\usepackage{caption}
\captionsetup{labelfont={color=blue}, textfont={color=blue}}
%%%%%%%%%%%%%%%%%%%%%%%%%%%%%%%%%%%%%%%%%%%%%%%%%%%%%%%%%%%%%%%%%%%%%%%%%%%%%%%%%%%%%%%%%%%%%%%%%%%%%%%%%%%%%%
\title{Universidad Panamericana \\ Maestría en Ciencia de Datos \\ Econometría \\ \vspace{0.5cm} Actividad RLM}
\author{Enrique Ulises Báez Gómez Tagle}
\date{\today}
%%%%%%%%%%%%%%%%%%%%%%%%%%%%%%%%%%%%%%%%%%%%%%%%%%%%%%%%%%%%%%%%%%%%%%%%%%%%%%%%%%%%%%%%%%%%%%%%%%%%%%%%%%%%%%
\begin{document}
\maketitle
\tableofcontents
\newpage
%%%%%%%%%%%%%%%%%%%%%%%%%%%%%%%%%%%%%%%%%%%%%%%%%%%%%%%%%%%%%%%%%%%%%%%%%%%%%%%%%%%%%%%%%%%%%%%%%%%%%%%%%%%%%%
\section{Introducción}
Se presentan y analizan el siguiente juego de datos cuyas variables son:
\begin{table}[H]
    \centering
    \captionsetup{labelfont={color=black}, textfont={color=black}}
    \begin{tabular}{|c|l|}
        \hline
        \textbf{Variable} & \textbf{Descripción} \\
        \hline
        1 Educación & Gasto per cápita en educación pública (en dólares) \\
        \hline
        2 Ingreso & Ingreso per cápita anual (en dólares) \\
        \hline
        3 Menores & Porcentaje de menores de 18 años de edad (por cada mil) \\
        \hline
        4 Urbano & Proporción de la población que reside en áreas urbanas \\
        \hline
    \end{tabular}
    \caption{Variables del conjuto de datos}
\end{table}
A continuación se presentan las primeras filas del conjunto de datos:
\begin{table}
\caption{Primeras filas del conjunto de datos utilizado en el análisis}
\label{tab:head_datos}
\begin{tabular}{lrrrr}
\toprule
Estado & educacion & Ingreso & Menores & Urbano \\
\midrule
ME & 189 & 2824 & 350.7000 & 508 \\
NH & 169 & 3259 & 345.9000 & 564 \\
VT & 230 & 3072 & 348.5000 & 322 \\
MA & 168 & 3835 & 335.3000 & 846 \\
RI & 180 & 3549 & 327.1000 & 871 \\
\bottomrule
\end{tabular}
\end{table}

%%%%%%%%%%%%%%%%%%%%%%%%%%%%%%%%%%%%%%%%%%%%%%%%%%%%%%%%%%%%%%%%%%%%%%%%%%%%%%%%%%%%%%%%%%%%%%%%%%%%%%%%%%%%%%
\section{Pregunta 1}
Utilizando los datos, considere el modelo de regresión lineal múltiple 
\[
y = \beta_0 + \beta_1 x_1 + \beta_2 x_2 + \beta_3 x_3 + \varepsilon
\]
donde $y$ representa la respuesta educación, $x_1$ el ingreso per cápita, $x_2$ el porcentaje de menores de 18 años y $x_3$ la proporción de habitantes que reside en áreas urbanas. 
Realice el ajuste del modelo (1). \\

    \textcolor{blue}{
        \textbf{Ajuste por Mínimos Cuadrados Ordinarios (MCO):} El estimador viene dado por:
        \[
        \hat{\boldsymbol{\beta}}=(\mathbf{X}'\mathbf{X})^{-1}\mathbf{X}'\mathbf{y},\quad \hat{\mathbf{y}}=\mathbf{X}\hat{\boldsymbol{\beta}}.
        \]
        Usando los datos, el modelo estimado es:
        \[
        \widehat{y}= -286.8388\; +\; 0.080653\, x_1\; +\; 0.817338\, x_2\; -\; 0.105806\, x_3,
        \]
        \noindent donde $x_1$ es \emph{ingreso}, $x_2$ \emph{menores} y $x_3$ \emph{urbano}. \\
        \noindent \textbf{Coeficientes y errores estándar:}
        \begin{table}[H]
    \centering
    \color{blue}
    \caption{Coeficientes del modelo OLS: Educación vs Ingreso, Menores y Urbano}
    \label{tab:ols_coef_educacion}
    \begin{tabular}{lrrrrrr}
        \toprule
            Parámetro & Coef. & Err. Std. & t & p & IC 2.5\% & IC 97.5\% \\
        \midrule
            Intercepto & -286.8388 & 64.9199 & -4.4183 & 0.0001 & -417.4408 & -156.2367 \\
            ingreso & 0.0807 & 0.0093 & 8.6738 & 0.0000 & 0.0619 & 0.0994 \\
            menores & 0.8173 & 0.1598 & 5.1151 & 0.0000 & 0.4959 & 1.1388 \\
            urbano & -0.1058 & 0.0343 & -3.0863 & 0.0034 & -0.1748 & -0.0368 \\
        \bottomrule
    \end{tabular}
\end{table}

        \noindent \textbf{Métricas del ajuste:} $R^2=0.690$, $R^2_{aj}=0.670$, $F(3,47)=34.81$, $\,\,p\text{-valor}<10^{-10}$.\\
        \begin{table}[H]
    \centering
    \color{blue}
    \caption{Métricas globales del modelo de regresión lineal múltiple}
    \label{tab:metricas_regresion}
    \begin{tabular}{lr}
        
            Métrica & Valor \\
        
            R² & 0.6896 \\
            R² ajustado & 0.6698 \\
            Estadístico F & 34.8105 \\
            p-valor (F) & 0.0000 \\
            AIC & 483.5767 \\
            BIC & 491.3040 \\
            Observaciones & 51.0000 \\
        
    \end{tabular}
\end{table}

    }
%%%%%%%%%%%%%%%%%%%%%%%%%%%%%%%%%%%%%%%%%%%%%%%%%%%%%%%%%%%%%%%%%%%%%%%%%%%%%%%%%%%%%%%%%%%%%%%%%%%%%%%%%%%%%%
\section{Pregunta 2}
Encuentre una estimación de la varianza de los errores $S^2 = e'e/n$, la matriz de covarianzas del vector de parámetros y los errores estándar de los coeficientes individuales. \\

\textcolor{blue}{RESPUESTA: }
%%%%%%%%%%%%%%%%%%%%%%%%%%%%%%%%%%%%%%%%%%%%%%%%%%%%%%%%%%%%%%%%%%%%%%%%%%%%%%%%%%%%%%%%%%%%%%%%%%%%%%%%%%%%%%
\section{Pregunta 3}
Construya un intervalo del 90\% de confianza para el coeficiente $\beta_2$. \\

\textcolor{blue}{RESPUESTA: }
%%%%%%%%%%%%%%%%%%%%%%%%%%%%%%%%%%%%%%%%%%%%%%%%%%%%%%%%%%%%%%%%%%%%%%%%%%%%%%%%%%%%%%%%%%%%%%%%%%%%%%%%%%%%%%
\section{Pregunta 4}
Calcule el gasto en educación pública que se esperaría a un nivel “promedio” de los regresores, esto es $(1,\bar{x})$. \\

\textcolor{blue}{RESPUESTA: }
%%%%%%%%%%%%%%%%%%%%%%%%%%%%%%%%%%%%%%%%%%%%%%%%%%%%%%%%%%%%%%%%%%%%%%%%%%%%%%%%%%%%%%%%%%%%%%%%%%%%%%%%%%%%%%
\section{Pregunta 5}
Realice la prueba de significancia del modelo de regresión (1), indicando claramente la hipótesis, estadístico de prueba, región de rechazo y conclusión. \\

\textcolor{blue}{RESPUESTA: }
%%%%%%%%%%%%%%%%%%%%%%%%%%%%%%%%%%%%%%%%%%%%%%%%%%%%%%%%%%%%%%%%%%%%%%%%%%%%%%%%%%%%%%%%%%%%%%%%%%%%%%%%%%%%%%
\section{Link al repositorio con código fuente}
\url{https://github.com/enriquegomeztagle/MCD-Econometria/tree/main/HWs/MLR-practice}
%%%%%%%%%%%%%%%%%%%%%%%%%%%%%%%%%%%%%%%%%%%%%%%%%%%%%%%%%%%%%%%%%%%%%%%%%%%%%%%%%%%%%%%%%%%%%%%%%%%%%%%%%%%%%%
\end{document}

