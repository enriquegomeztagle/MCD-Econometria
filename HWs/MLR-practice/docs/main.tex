\documentclass[10pt]{article}

\usepackage[utf8]{inputenc}
\usepackage[spanish]{babel}
\decimalpoint
\usepackage{amsmath, amssymb}
\usepackage{xcolor}
\usepackage{geometry}
\geometry{letterpaper, margin=1in}
\usepackage{graphicx}
\usepackage{hyperref}
\usepackage{float}
\usepackage{colortbl}
\usepackage{caption}
\captionsetup{labelfont={color=blue}, textfont={color=blue}}
%%%%%%%%%%%%%%%%%%%%%%%%%%%%%%%%%%%%%%%%%%%%%%%%%%%%%%%%%%%%%%%%%%%%%%%%%%%%%%%%%%%%%%%%%%%%%%%%%%%%%%%%%%%%%%
\title{Universidad Panamericana \\ Maestría en Ciencia de Datos \\ Econometría \\ \vspace{0.5cm} Actividad RLM}
\author{Enrique Ulises Báez Gómez Tagle}
\date{\today}
%%%%%%%%%%%%%%%%%%%%%%%%%%%%%%%%%%%%%%%%%%%%%%%%%%%%%%%%%%%%%%%%%%%%%%%%%%%%%%%%%%%%%%%%%%%%%%%%%%%%%%%%%%%%%%
\begin{document}
\maketitle
\tableofcontents
\newpage
%%%%%%%%%%%%%%%%%%%%%%%%%%%%%%%%%%%%%%%%%%%%%%%%%%%%%%%%%%%%%%%%%%%%%%%%%%%%%%%%%%%%%%%%%%%%%%%%%%%%%%%%%%%%%%
\section{Introducción}
Se presentan y analizan el siguiente juego de datos cuyas variables son:
\begin{table}[H]
    \centering
    \captionsetup{labelfont={color=black}, textfont={color=black}}
    \begin{tabular}{|c|l|}
        \hline
        \textbf{Variable} & \textbf{Descripción} \\
        \hline
        1 Educación & Gasto per cápita en educación pública (en dólares) \\
        \hline
        2 Ingreso & Ingreso per cápita anual (en dólares) \\
        \hline
        3 Menores & Porcentaje de menores de 18 años de edad (por cada mil) \\
        \hline
        4 Urbano & Proporción de la población que reside en áreas urbanas \\
        \hline
    \end{tabular}
    \caption{Variables del conjuto de datos}
\end{table}
A continuación se presentan las primeras filas del conjunto de datos:
\begin{table}
\caption{Primeras filas del conjunto de datos utilizado en el análisis}
\label{tab:head_datos}
\begin{tabular}{lrrrr}
\toprule
Estado & educacion & Ingreso & Menores & Urbano \\
\midrule
ME & 189 & 2824 & 350.7000 & 508 \\
NH & 169 & 3259 & 345.9000 & 564 \\
VT & 230 & 3072 & 348.5000 & 322 \\
MA & 168 & 3835 & 335.3000 & 846 \\
RI & 180 & 3549 & 327.1000 & 871 \\
\bottomrule
\end{tabular}
\end{table}

%%%%%%%%%%%%%%%%%%%%%%%%%%%%%%%%%%%%%%%%%%%%%%%%%%%%%%%%%%%%%%%%%%%%%%%%%%%%%%%%%%%%%%%%%%%%%%%%%%%%%%%%%%%%%%
\section{Pregunta 1}
Utilizando los datos, considere el modelo de regresión lineal múltiple 
\[
y = \beta_0 + \beta_1 x_1 + \beta_2 x_2 + \beta_3 x_3 + \varepsilon
\]
donde $y$ representa la respuesta educación, $x_1$ el ingreso per cápita, $x_2$ el porcentaje de menores de 18 años y $x_3$ la proporción de habitantes que reside en áreas urbanas. 
Realice el ajuste del modelo (1). \\

    \textcolor{blue}{
        \textbf{Ajuste por Mínimos Cuadrados Ordinarios (MCO):} El estimador viene dado por:
        \[
        \hat{\boldsymbol{\beta}}=(\mathbf{X}'\mathbf{X})^{-1}\mathbf{X}'\mathbf{y},\quad \hat{\mathbf{y}}=\mathbf{X}\hat{\boldsymbol{\beta}}.
        \]
        Usando los datos, el modelo estimado es:
        \[
        \widehat{y}= -286.8388\; +\; 0.080653\, x_1\; +\; 0.817338\, x_2\; -\; 0.105806\, x_3,
        \]
        \noindent donde $x_1$ es \emph{ingreso}, $x_2$ \emph{menores} y $x_3$ \emph{urbano}. \\
        \noindent \textbf{Coeficientes y errores estándar:}
        \begin{table}[H]
    \centering
    \color{blue}
    \caption{Coeficientes del modelo OLS: Educación vs Ingreso, Menores y Urbano}
    \label{tab:ols_coef_educacion}
    \begin{tabular}{lrrrrrr}
        \toprule
            Parámetro & Coef. & Err. Std. & t & p & IC 2.5\% & IC 97.5\% \\
        \midrule
            Intercepto & -286.8388 & 64.9199 & -4.4183 & 0.0001 & -417.4408 & -156.2367 \\
            ingreso & 0.0807 & 0.0093 & 8.6738 & 0.0000 & 0.0619 & 0.0994 \\
            menores & 0.8173 & 0.1598 & 5.1151 & 0.0000 & 0.4959 & 1.1388 \\
            urbano & -0.1058 & 0.0343 & -3.0863 & 0.0034 & -0.1748 & -0.0368 \\
        \bottomrule
    \end{tabular}
\end{table}

        \noindent \textbf{Métricas del ajuste:} $R^2=0.690$, $R^2_{aj}=0.670$, $F(3,47)=34.81$, $\,\,p\text{-valor}<10^{-10}$.\\
        \begin{table}[H]
    \centering
    \color{blue}
    \caption{Métricas globales del modelo de regresión lineal múltiple}
    \label{tab:metricas_regresion}
    \begin{tabular}{lr}
        
            Métrica & Valor \\
        
            R² & 0.6896 \\
            R² ajustado & 0.6698 \\
            Estadístico F & 34.8105 \\
            p-valor (F) & 0.0000 \\
            AIC & 483.5767 \\
            BIC & 491.3040 \\
            Observaciones & 51.0000 \\
        
    \end{tabular}
\end{table}

    }
%%%%%%%%%%%%%%%%%%%%%%%%%%%%%%%%%%%%%%%%%%%%%%%%%%%%%%%%%%%%%%%%%%%%%%%%%%%%%%%%%%%%%%%%%%%%%%%%%%%%%%%%%%%%%%
\section{Pregunta 2}
Encuentre una estimación de la varianza de los errores $S^2 = e'e/n$, la matriz de covarianzas del vector de parámetros y los errores estándar de los coeficientes individuales. \\

\textcolor{blue}{
    \textbf{Estimación de la varianza del error:}
    Sea $\hat{\mathbf e}=\mathbf y-\mathbf X\hat{\boldsymbol\beta}$. La varianza de los errores se estima como
    \[
        S^2 = \frac{\hat{\mathbf e}'\hat{\mathbf e}}{n}
        \quad \text{(} \widehat{\sigma}^2 = \tfrac{\hat{\mathbf e}'\hat{\mathbf e}}{n-p} \text{ con } p=4\text{)}.
    \]
    Con los datos del ejercicio se obtiene:
    \begin{table}[H]
    \centering
    \color{blue}
    \caption{Estimación de la varianza del error $S^2$ del modelo OLS.}
    \label{tab:varianza_error}
    \begin{tabular}{lr}
        \toprule
        Métrica & Valor \\
        \midrule
        $S^2$ (Varianza del error) & 712.5394 \\
        \bottomrule
    \end{tabular}
\end{table}

    \textbf{Matriz de covarianzas del vector de parámetros} $\hat{\boldsymbol\beta}$.
    Por teoría MCO,
    \[
        \widehat{\operatorname{Var}}(\hat{\boldsymbol\beta}) = S^2 (\mathbf X'\mathbf X)^{-1}.
    \]
    Su estimación numérica es:
    \begin{table}[H]
    \centering
    \color{blue}
    \caption{Matriz de covarianzas de los estimadores del modelo OLS.}
    \label{tab:covarianzas_parametros}
    \begin{tabular}{lrrrr}
        \toprule
        Parámetro & Intercepto & ingreso & menores & urbano \\
        \midrule
        Intercepto & 4214.5975 & -0.1850 & -9.7494 & -0.1583 \\
        ingreso   & -0.1850   & 0.0001  & 0.0001  & -0.0002 \\
        menores   & -9.7494   & 0.0001  & 0.0255  & 0.0002  \\
        urbano    & -0.1583   & -0.0002 & 0.0002  & 0.0012  \\
        \bottomrule
    \end{tabular}
\end{table}

    \textbf{Errores estándar de los coeficientes individuales.}
    Los errores estándar son las raíces cuadradas de la diagonal de $\widehat{\operatorname{Var}}(\hat{\boldsymbol\beta})$. A continuación se reportan junto con los coeficientes:
    \begin{table}[H]
    \centering
    \color{blue}
    \caption{Estimaciones, errores estándar y pruebas t de los coeficientes del modelo}
    \label{tab:coeficientes_regresion}
    \begin{tabular}{rrrr}
        
        Coeficiente & Error Estándar & t & p \\
        
        -286.8388 & 64.9199 & -4.4183 & 0.0001 \\
        0.0807    & 0.0093  &  8.6738 & 0.0000 \\
        0.8173    & 0.1598  &  5.1151 & 0.0000 \\
        -0.1058   & 0.0343  & -3.0863 & 0.0034 \\
        
    \end{tabular}
\end{table}

}
%%%%%%%%%%%%%%%%%%%%%%%%%%%%%%%%%%%%%%%%%%%%%%%%%%%%%%%%%%%%%%%%%%%%%%%%%%%%%%%%%%%%%%%%%%%%%%%%%%%%%%%%%%%%%%
\section{Pregunta 3}
Construya un intervalo del 90\% de confianza para el coeficiente $\beta_2$. \\

\textcolor{blue}{
    \textbf{IC del 90\% para $\beta_2$ (Menores):}
    Sea $\hat\beta_2$ el estimador de $\beta_2$ y $\operatorname{EE}(\hat\beta_2)$ su error estándar. El intervalo de confianza bilateral al 90\% está dado por
    \[
        \hat\beta_2 \;\pm\; t_{1-\alpha/2,\,n-p}\; \operatorname{EE}(\hat\beta_2),\quad \alpha=0.10,\; n=51,\; p=4 \Rightarrow \text{gl}=n-p=47.
    \]
    A partir del ajuste MCO del inciso anterior (ver tabla de coeficientes) tenemos $\hat\beta_2=0.817338$ y $\operatorname{EE}(\hat\beta_2)=0.159790$. Usando $t_{0.95,47}\approx 1.677$ se obtiene el intervalo numérico siguiente:
    \begin{table}
\caption{Intervalo de confianza del 90\% para el coeficiente $\beta_2$ (Ingreso).}
\label{tab:ci_beta2}
\begin{tabular}{lrr}
\toprule
Parámetro & Límite inferior & Límite superior \\
\midrule
ingreso & 0.0651 & 0.0963 \\
\bottomrule
\end{tabular}
\end{table}

    Para referencia, volvemos a listar los coeficientes estimados y sus errores estándar:
    \begin{table}[H]
    \centering
    \color{blue}
    \caption{Coeficientes del modelo OLS: Educación vs Ingreso, Menores y Urbano}
    \label{tab:ols_coef_educacion}
    \begin{tabular}{lrrrrrr}
        \toprule
            Parámetro & Coef. & Err. Std. & t & p & IC 2.5\% & IC 97.5\% \\
        \midrule
            Intercepto & -286.8388 & 64.9199 & -4.4183 & 0.0001 & -417.4408 & -156.2367 \\
            ingreso & 0.0807 & 0.0093 & 8.6738 & 0.0000 & 0.0619 & 0.0994 \\
            menores & 0.8173 & 0.1598 & 5.1151 & 0.0000 & 0.4959 & 1.1388 \\
            urbano & -0.1058 & 0.0343 & -3.0863 & 0.0034 & -0.1748 & -0.0368 \\
        \bottomrule
    \end{tabular}
\end{table}

}
%%%%%%%%%%%%%%%%%%%%%%%%%%%%%%%%%%%%%%%%%%%%%%%%%%%%%%%%%%%%%%%%%%%%%%%%%%%%%%%%%%%%%%%%%%%%%%%%%%%%%%%%%%%%%%
\section{Pregunta 4}
Calcule el gasto en educación pública que se esperaría a un nivel “promedio” de los regresores, esto es $(1,\bar{x})$. \\

\textcolor{blue}{
    \textbf{Predicción en el punto promedio $(1,\bar x)$:}
    Dado que la predicción en un punto específico $\mathbf{x}_0 = (1,\bar x_1, \bar x_2, \bar x_3)'$ viene dada por:
    \[
        \hat y(\mathbf{x}_0) = \mathbf{x}_0' \hat{\boldsymbol\beta} 
        = \hat\beta_0 + \hat\beta_1\,\bar x_1 + \hat\beta_2\,\bar x_2 + \hat\beta_3\,\bar x_3.
    \]
    Los promedios muestrales de los regresores son:
    \begin{table}
\caption{Valores promedio de los regresores y gasto en educación esperado en el punto promedio}
\label{tab:prediccion_media}
\begin{tabular}{lr}
\toprule
Variable & Valor promedio \\
\midrule
Ingreso & 3225.2900 \\
Menores & 358.8900 \\
Urbano & 664.5100 \\
Educación esperada & 196.3100 \\
\bottomrule
\end{tabular}
\end{table}

    Sustituyendo estos valores en la ecuación estimada se obtiene el gasto en educación pública esperado en el nivel promedio de los regresores:
    \[
        \hat y(\bar x) = 196.31.
    \]
}

%%%%%%%%%%%%%%%%%%%%%%%%%%%%%%%%%%%%%%%%%%%%%%%%%%%%%%%%%%%%%%%%%%%%%%%%%%%%%%%%%%%%%%%%%%%%%%%%%%%%%%%%%%%%%%
\section{Pregunta 5}
Realice la prueba de significancia del modelo de regresión (1), indicando claramente la hipótesis, estadístico de prueba, región de rechazo y conclusión. \\

\textcolor{blue}{
    \textbf{Prueba de significancia global del modelo (Prueba F).}
    \textbf{Hipótesis:}\\[-2mm]
    \[
        H_0: \beta_1=\beta_2=\beta_3=0 \qquad \text{vs.} \qquad H_1: \text{al menos uno } \beta_j\neq 0.
    \]
    \textbf{Estadístico de prueba:} bajo $H_0$, el estadístico
    \[
        F\_{obs} 
        = \frac{\text{SSR}/k}{\text{SSE}/(n-p)}
        = \frac{(R^2/k)}{((1-R^2)/(n-p))}
        \;\sim\; F\_{k,\,n-p},
    \]
    donde $k=3$ es el número de regresores, $p=k+1=4$ el número de parámetros (incluye intercepto) y $n=51$ el tamaño de muestra. Con los datos del ajuste se obtuvo:
    \[
        F\_{obs}=34.8105, \quad (\mathrm{gl}_1,\mathrm{gl}_2)=(3,47).
    \]
    \textbf{Región de rechazo (nivel $\alpha=0.05$): } Rechazar $H_0$ si 
    \[
        F\_{obs} > F\_{0.95;\,3,47} = 2.8024.
    \]
    \textbf{Decisión por valor crítico:} como $34.8105 > 2.8024$, \textbf{se rechaza} $H_0$.\\
    \textbf{Decisión por $p$-valor:} $p$-valor $= 5.34\times 10^{-12} < 0.05$ \;$\Rightarrow$\; \textbf{se rechaza} $H_0$.\\
    \textbf{Conclusión:} el modelo de regresión es \emph{globalmente significativo}; al menos una de las variables explicativas (Ingreso, Menores, Urbano) contribuye de forma estadísticamente significativa a explicar el gasto en educación.\\
    \vspace{0.5cm}
    \noindent\textbf{Resumen numérico:}
    \begin{table}[H]
    \centering
    \color{blue}
    \caption{Prueba de significancia global del modelo de regresión (estadístico F).}
    \label{tab:prueba_f}
    \begin{tabular}{ll}
        
        Estadístico & Valor \\
        
        F observado        & 34.8110 \\
        F crítico ($\alpha=0.05$) & 2.8020 \\
        p-valor            & 5.34e-12 \\
        gl1                & 3 \\
        gl2                & 47 \\
        
    \end{tabular}
\end{table}

    \begin{table}
\caption{Hipótesis y conclusión de la prueba F global del modelo de regresión}
\label{tab:hipotesis_conclusion}
\begin{tabular}{ll}
\toprule
Elemento & Descripción \\
\midrule
Hipótesis nula (H0) & $\beta_1 = \beta_2 = \beta_3 = 0$ (el modelo no es significativo) \\
Hipótesis alternativa (H1) & Al menos un $\beta_j \neq 0$ \\
Conclusión & Se RECHAZA H0: el modelo es globalmente significativo. \\
\bottomrule
\end{tabular}
\end{table}

}
%%%%%%%%%%%%%%%%%%%%%%%%%%%%%%%%%%%%%%%%%%%%%%%%%%%%%%%%%%%%%%%%%%%%%%%%%%%%%%%%%%%%%%%%%%%%%%%%%%%%%%%%%%%%%%
\section{Link al repositorio con código fuente}
\url{https://github.com/enriquegomeztagle/MCD-Econometria/tree/main/HWs/MLR-practice}
%%%%%%%%%%%%%%%%%%%%%%%%%%%%%%%%%%%%%%%%%%%%%%%%%%%%%%%%%%%%%%%%%%%%%%%%%%%%%%%%%%%%%%%%%%%%%%%%%%%%%%%%%%%%%%
\end{document}
