\documentclass[10pt]{article}

\usepackage[utf8]{inputenc}
\usepackage[spanish]{babel}
\decimalpoint
\usepackage{amsmath, amssymb}
\usepackage{xcolor}
\usepackage{geometry}
\geometry{letterpaper, margin=1in}
\usepackage{graphicx}
\usepackage{hyperref}
\usepackage{float}
\usepackage{colortbl}
\usepackage{caption}
%%%%%%%%%%%%%%%%%%%%%%%%%%%%%%%%%%%%%%%%%%%%%%%%%%%%%%%%%%%%%%%%%%%%%%%%%%%%%%%%%%%%%%%%%%%%%%%%%%%%%%%%%%%%%%
\title{Universidad Panamericana \\ Maestría en Ciencia de Datos \\ Econometría \\ \vspace{0.5cm} Actividad RLM}
\author{Enrique Ulises Báez Gómez Tagle}
\date{\today}
%%%%%%%%%%%%%%%%%%%%%%%%%%%%%%%%%%%%%%%%%%%%%%%%%%%%%%%%%%%%%%%%%%%%%%%%%%%%%%%%%%%%%%%%%%%%%%%%%%%%%%%%%%%%%%
\begin{document}
\maketitle
\tableofcontents
\newpage
%%%%%%%%%%%%%%%%%%%%%%%%%%%%%%%%%%%%%%%%%%%%%%%%%%%%%%%%%%%%%%%%%%%%%%%%%%%%%%%%%%%%%%%%%%%%%%%%%%%%%%%%%%%%%%
\section{Introducción}
Se presentan y analizan el siguiente juego de datos cuyas variables son:
\begin{table}[H]
\centering
\begin{tabular}{|c|l|}
\hline
\textbf{Variable} & \textbf{Descripción} \\
\hline
1 Educación & Gasto per cápita en educación pública (en dólares) \\
\hline
2 Ingreso & Ingreso per cápita anual (en dólares) \\
\hline
3 Menores & Porcentaje de menores de 18 años de edad (por cada mil) \\
\hline
4 Urbano & Proporción de la población que reside en áreas urbanas \\
\hline
\end{tabular}
\caption{Variables del conjuto de datos}
\end{table}
%%%%%%%%%%%%%%%%%%%%%%%%%%%%%%%%%%%%%%%%%%%%%%%%%%%%%%%%%%%%%%%%%%%%%%%%%%%%%%%%%%%%%%%%%%%%%%%%%%%%%%%%%%%%%%
\section{Pregunta 1}
Utilizando los datos, considere el modelo de regresión lineal múltiple 
\[
y = \beta_0 + \beta_1 x_1 + \beta_2 x_2 + \beta_3 x_3 + \varepsilon
\]
donde $y$ representa la respuesta educación, $x_1$ el ingreso per cápita, $x_2$ el porcentaje de menores de 18 años y $x_3$ la proporción de habitantes que reside en áreas urbanas. Realice el ajuste del modelo (1). \\

\textcolor{blue}{RESPUESTA: }
%%%%%%%%%%%%%%%%%%%%%%%%%%%%%%%%%%%%%%%%%%%%%%%%%%%%%%%%%%%%%%%%%%%%%%%%%%%%%%%%%%%%%%%%%%%%%%%%%%%%%%%%%%%%%%
\section{Pregunta 2}
Encuentre una estimación de la varianza de los errores $S^2 = e'e/n$, la matriz de covarianzas del vector de parámetros y los errores estándar de los coeficientes individuales. \\

\textcolor{blue}{RESPUESTA: }
%%%%%%%%%%%%%%%%%%%%%%%%%%%%%%%%%%%%%%%%%%%%%%%%%%%%%%%%%%%%%%%%%%%%%%%%%%%%%%%%%%%%%%%%%%%%%%%%%%%%%%%%%%%%%%
\section{Pregunta 3}
Construya un intervalo del 90\% de confianza para el coeficiente $\beta_2$. \\

\textcolor{blue}{RESPUESTA: }
%%%%%%%%%%%%%%%%%%%%%%%%%%%%%%%%%%%%%%%%%%%%%%%%%%%%%%%%%%%%%%%%%%%%%%%%%%%%%%%%%%%%%%%%%%%%%%%%%%%%%%%%%%%%%%
\section{Pregunta 4}
Calcule el gasto en educación pública que se esperaría a un nivel “promedio” de los regresores, esto es $(1,\bar{x})$. \\

\textcolor{blue}{RESPUESTA: }
%%%%%%%%%%%%%%%%%%%%%%%%%%%%%%%%%%%%%%%%%%%%%%%%%%%%%%%%%%%%%%%%%%%%%%%%%%%%%%%%%%%%%%%%%%%%%%%%%%%%%%%%%%%%%%
\section{Pregunta 5}
Realice la prueba de significancia del modelo de regresión (1), indicando claramente la hipótesis, estadístico de prueba, región de rechazo y conclusión. \\

\textcolor{blue}{RESPUESTA: }
%%%%%%%%%%%%%%%%%%%%%%%%%%%%%%%%%%%%%%%%%%%%%%%%%%%%%%%%%%%%%%%%%%%%%%%%%%%%%%%%%%%%%%%%%%%%%%%%%%%%%%%%%%%%%%
\section{Link al repositorio con código fuente}
\url{https://github.com/enriquegomeztagle/MCD-Econometria/tree/main/HWs/MLR-practice}
%%%%%%%%%%%%%%%%%%%%%%%%%%%%%%%%%%%%%%%%%%%%%%%%%%%%%%%%%%%%%%%%%%%%%%%%%%%%%%%%%%%%%%%%%%%%%%%%%%%%%%%%%%%%%%
\end{document}

