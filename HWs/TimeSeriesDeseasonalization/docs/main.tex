\documentclass[10pt]{article}

\usepackage[utf8]{inputenc}
\usepackage[spanish]{babel}
\decimalpoint
\usepackage{amsmath, amssymb}
\usepackage{xcolor}
\usepackage{geometry}
\geometry{letterpaper, margin=1in}
\usepackage{graphicx}
\usepackage{hyperref}
\usepackage{float}
\usepackage{colortbl}
\usepackage{caption}


\title{Universidad Panamericana \\ Maestría en Ciencia de Datos 
\\ Econometría \\ \vspace{0.5cm} Actividad Desestacionalización de Series de Tiempo}
\author{Enrique Ulises Báez Gómez Tagle}
\date{\today}

\begin{document}

\maketitle

\tableofcontents

\newpage


\section{Contexto}
México es el mayor productor y exportador mundial de aguacate, una fruta que ha ganado popularidad global 
por su valor nutricional y versatilidad en la cocina. El aguacate, conocido por su riqueza en grasas 
saludables, vitaminas y minerales, es un componente esencial en dietas equilibradas y ha visto un aumento 
significativo en su demanda en mercados internacionales.

La industria del aguacate es un pilar fundamental de la economía agrícola mexicana. En las últimas 
décadas, el aguacate ha pasado de ser un producto regional a convertirse en una de las principales 
exportaciones agrícolas del país. La producción y exportación de aguacate no solo contribuyen de manera 
significativa al PIB agrícola, sino que también generan empleo en diversas regiones, particularmente en 
los estados de Michoacán y Jalisco, que son los principales productores.

A continuación en la tabla no1 se muestran los datos bimestrales de la exportación de aguacate en toneladas:

\begin{table}[H]
\centering
\caption{Tabla no 1: Exportación bimestral de aguacate (toneladas)}
\label{tab:export_avocado}
\begin{tabular}{l c c r}
\hline
\textbf{Año} & \textbf{Bimestre} & \textbf{t} & \textbf{Toneladas} \\
\hline
2019 & 1 & 1  & 224,604.04 \\
2019 & 2 & 2  & 201,125.90 \\
2019 & 3 & 3  & 154,184.06 \\
2019 & 4 & 4  & 140,857.77 \\
2019 & 5 & 5  & 213,070.50 \\
2019 & 6 & 6  & 268,644.20 \\
2020 & 1 & 7  & 316,793.52 \\
2020 & 2 & 8  & 283,659.09 \\
2020 & 3 & 9  & 217,428.69 \\
2020 & 4 & 10 & 198,535.98 \\
2020 & 5 & 11 & 300,467.80 \\
2020 & 6 & 12 & 378,764.26 \\
2021 & 1 & 13 & 410,806.71 \\
2021 & 2 & 14 & 367,955.31 \\
2021 & 3 & 15 & 282,092.18 \\
2021 & 4 & 16 & 257,532.45 \\
2021 & 5 & 17 & 389,706.87 \\
2021 & 6 & 18 & 491,240.80 \\
2022 & 1 & 19 & 405,077.56 \\
2022 & 2 & 20 & 362,765.23 \\
2022 & 3 & 21 & 278,092.64 \\
2022 & 4 & 22 & 253,986.68 \\
2022 & 5 & 23 & 384,156.60 \\
2022 & 6 & 24 & 484,431.36 \\
2023 & 1 & 25 & 426,151.46 \\
2023 & 2 & 26 & 381,653.51 \\
2023 & 3 & 27 & 292,635.95 \\
2023 & 4 & 28 & 267,180.48 \\
2023 & 5 & 29 & 404,208.92 \\
2023 & 6 & 30 & 522,430.98 \\
\hline
\end{tabular}
\end{table}

\section{Link al repositorio con código fuente}
\href{https://github.com/enriquegomeztagle/MCD-Econometria/tree/main/HWs/TimeSeriesDeseasonalization}{https://github.com/enriquegomeztagle/MCD-Econometria/tree/main/HWs/TimeSeriesDeseasonalization}

\end{document}

%%%%%%
